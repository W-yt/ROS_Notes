\documentclass[9pt, oneside]{book}
\usepackage{xeCJK}
\usepackage{amsmath, amsthm, amssymb, bm, graphicx, hyperref, mathrsfs}
\usepackage{geometry}
% \geometry{b5paper,scale=0.85}
\geometry{b5paper,left=1.2cm,right=1.2cm,top=2cm,bottom=1cm}
\usepackage{graphicx} %插入图片的宏包
\usepackage{float} %设置图片浮动位置的宏包
\usepackage{subfigure} %插入多图时用子图显示的宏包
\usepackage{amstext} %公式中包含文字的宏包
\usepackage{booktabs} %插入表格的宏包
\usepackage{multirow} 
\usepackage{indentfirst} %设置缩进的宏包
\setlength{\parindent}{2em}
\usepackage{enumerate} %用于编号的宏包
\usepackage{hyperref} %用于引用的宏包
% \hypersetup{colorlinks, linkcolor=blue} %设置引用的字体颜色
\usepackage{color} %用于设置字体颜色的宏包
\usepackage{url} %用于超链接的宏包


% 封面部分
\title{\Huge{\textbf{ROS Notebook}}}
\author{Wu Yutian}
\date{2021.11.13}
\linespread{1.4}
\newtheorem{theorem}{定理}[section]
\newtheorem{definition}[theorem]{定义}
\newtheorem{lemma}[theorem]{引理}
\newtheorem{corollary}[theorem]{推论}
\newtheorem{example}[theorem]{例}
\newtheorem{proposition}[theorem]{命题}
\begin{document}

% 输出封面
\maketitle

% 前言部分
\pagenumbering{roman}
\setcounter{page}{1}

\begin{center}
    \Huge\textbf{前言}
\end{center}~\

\noindent{\large{本书的主要内容包括:}}
\normalsize
\begin{itemize}
    \item [-] 学习古月居的相关入门课程视频的内容记录
    \item [-] 阅读胡春旭的《ROS机器人开发实践》的笔记整理
    \item [-] 参考高翔的《视觉SLAM十四讲》补充了关于三维刚体运动学的内容
    \item [-] 参考一些博客阅读ros-navigation导航包源码的思路整理
    \item [-]
\end{itemize}

~\\
\begin{flushright}     
    \begin{tabular}{c}
        Wu Yutian\\
        2021.11.13
    \end{tabular}
\end{flushright}

\newpage
\pagenumbering{Roman}
\setcounter{page}{1}
\tableofcontents
\newpage
\setcounter{page}{1}
\pagenumbering{arabic}





\chapter{Navigation详细学习}

\section{base\_local\_planner源码学习}



\subsection{源码相关文件}

\begin{itemize}
    \item 源码链接:
    
    \url{https://github.com/ros-planning/navigation/tree/melodic-devel}

    \item 源码注释链接:
    
    \small
    \url{https://github.com/W-yt/ROS_Notes/tree/master/navigation-melodic-devel/base_local_planner}
    \normalsize
\end{itemize}

对应源码中的相关文件:

\begin{itemize}
    \item [-] 
    \item [-] 
    \item [-] 
    \item [-] 
    \item [-] 
    \item [-] 
    \item [-] 
    \item [-] 
\end{itemize}

\subsection{整体结构图}

% \begin{figure}[H]
%     \centering
%     \includegraphics[width=1.0\linewidth]{image/base_local_planner.png}
% \end{figure}

\subsection{参数配置}

参数配置文件为:$global\_planner\_params.yaml$

参数列表:

\begin{itemize}
    \item [-] 
\end{itemize}
























































\end{document}